\documentclass[usenames,dvipsnames]{beamer}
\usepackage{comment}
\usetheme{CambridgeUS}
%\usepackage{macros_gpi}

%\setbeamertemplate{blocks}[rounded][shadow=false]
\usepackage{epsfig}
\usepackage[french]{babel}
\usepackage[utf8x]{inputenc}
\usepackage[T1]{fontenc}
\usepackage{algorithm}
\usepackage{stmaryrd}
%\usepackage{slashbox}
\usepackage{algorithmic}
\usepackage{multirow}   
\usepackage{pstricks}    
\usepackage{color}   
\usepackage{pifont}
\usepackage{supertabular}   
\usepackage{graphicx}
\usepackage{graphbox}
\usepackage{caption}
\usepackage{subcaption}
\usepackage{animate}
%\usepackage{pdfpc-commands}
%\usepackage{xmpmulti}
\captionsetup[figure]{labelformat=empty}
\usepackage{tikz}
\usetikzlibrary{shadows}
\usepackage{fontawesome}
%\newlength{\myheight}

\usepackage{xcolor}
\definecolor{orange-perp}{rgb}{1.0,0.412,0}
\definecolor{prune-saclay}{rgb}{0.388,0,0.235}
\definecolor{bleu-nice}{rgb}{0,0.686,0.843}
\setbeamercolor{author in head/foot}{bg=black,fg=white}
\setbeamercolor{title in head/foot}{fg=black,bg=white}
\setbeamercolor{frametitle}{fg=black,bg=white}
\setbeamercolor{date in head/foot}{bg=black,fg=white}
\setbeamercolor{section in head/foot}{bg=black,fg=white}
\setbeamercolor{subsection in head/foot}{fg=black,bg=white}
\definecolor{darkspringgreen}{rgb}{0.09, 0.45, 0.27}

\setbeamercolor{block body alerted}{bg=red!0.2,fg=black}
\setbeamercolor{block title alerted}{bg=red,fg=white}

\setbeamercolor{block body}{bg=black!0.2,fg=black}
\setbeamercolor{block title}{bg=black,fg=white}
%\setbeamercolor{block body}{bg=structure!10}
%\setbeamercolor{block title}{bg=structure!20}

\selectlanguage{french}
\setbeamercolor{footlinecolor}{fg=white,bg=black}
\renewcommand\footnoterule{{\color{black}\hrule height 0.5pt width \paperwidth}}


\usepackage{hyperref}
\hypersetup{
  %colorlinks   = true, %Colours links instead of ugly boxes
  %urlcolor     = blue, %Colour for external hyperlinks
  %linkcolor    = blue, %Colour of internal links
  %citecolor   = red %Colour of citations
}

%\setbeamercolor{itemize item}{fg=prune}
%\setbeamercolor{itemize subitem}{fg=prune}
%\setbeamercolor{itemize subsubitem}{fg=prune}

%\setbeamertemplate{itemize item}[ball]{fg=prune}
%\setbeamertemplate{itemize subitem}[ball]{fg=prune}
%\setbeamertemplate{itemize subsubitem}[triangle]


\setbeamertemplate{itemize item}{%
    \begin{tikzpicture}
        \shade[ball color=black!100!white] (0,0) circle (0.6ex);
    \end{tikzpicture}
}

\setbeamertemplate{itemize subitem}{%
    \begin{tikzpicture}
        \shade[ball color=black!100!white] (0,0) circle (0.6ex);
    \end{tikzpicture}
}


\setbeamercolor*{title}{use=structure,fg=white,bg=black!95}
\setbeamertemplate{title page}[default][colsep=-4bp,rounded=true,shadow=true]
\setbeamertemplate{navigation symbols}{} 
%\setbeamerfont{section number projected}{size=\footnotesize}
%\setbeamercolor{section number projected}{bg=prune,fg=white}
%\setbeamercolor{section in toc}{fg=prune}
%\setbeamercolor{subsection in toc}{fg=prune}
%\setbeamercolor{subsection number projected}{bg=prune}

\setbeamercolor{section number projected}{bg=black,fg=white}
\setbeamercolor{section in toc}{fg=black}
\setbeamercolor{subsection in toc}{fg=black}
\setbeamercolor{subsection number projected}{bg=black}

%\setbeamertemplate{subsections in toc}[square]
\mode<all>

%\setbeamertemplate{footline}{
%\begin{picture}(0,0)(0,0)
%\put(320,4){\footnotesize \insertframenumber{}/\inserttotalframenumber{}}
%\end{picture}
%}
\usepackage{hyperref}
\usepackage{siunitx}

%\includegraphics[width=1cm]{logo-projet-finance-par-ANR.jpg}
\title[Dark-era - Session\#1]{\textbf{Dark-era - Session\#1}} 
\subtitle{Dataflow Algorithm aRchitecture co-design of SKA pipeline for Exascale Radio Astronomy\\ }
\institute[]{Daniel Charlet$^{**5}$ (IJCLab), Karol Desnos$^{1}$, Mickael Dardaillon$^{3}$, André Ferrari$^{4}$, Chiara Ferrari$^{4}$, Nicolas Gac$^{3}$, Jean-François Nezan$^{1}$, François Orieux$^{3}$, Simon Prunet$^{4}$, Martin Quinson$^{2}$, Frédéric Suter$^{**2}$(IN2P3 Computing Center), Cyril Tasse$^{**5}$ (GEPI), Cédric Viou$^{5}$ }
\author[IETR/IRISA/L2S/Lagrange/Nançay]{$^{1}$IETR (INSA),  $^{2}$IRISA (ENS), $^{3}$L2S (CS), $^{4}$Lagrange (UCA),  $^{5}$Nançay (Obs Paris)}
\date{17 mai 2021, Teams}


\newcommand{\mysection}[2][blue]{%
    \begingroup
    \setbeamercolor{background canvas}{bg=#1}
    \setbeamercolor{section title}{fg=-#1}
    \section{#2}
    \endgroup
}

\begin{document}

\frame[plain]{\titlepage}

\section*{Programme session}
\setcounter{tocdepth}{1}
\frame{
    %\frametitle{Plan }
    \tableofcontents
    }
\setbeamerfont{section number projected}{size=\footnotesize}
\setcounter{tocdepth}{2}

\section{News}
\frame[noframenumbering]{\tableofcontents[currentsection]}

\subsection{Logo}

\frame{
\frametitle{.. and the winner is :}
\includegraphics[width=0.49\textwidth]{Restitution-PEPS/DARKERA_logo_color.pdf} 
\includegraphics[width=0.49\textwidth]{Restitution-PEPS/DARKERA_logo_monochrome.pdf} 
\begin{block}{Proposé par Cédric tuné par Karol}
 clin d'œil cosmologique en rapport aux ages sombres (dark era) qui arrivent juste après le CMB (Cosmic Microwave Background).
\end{block}
}


\frame{
\frametitle{Logo disponible sous le cloud}
\begin{block}{Merci à Karol pour la déclinaison sous différents formats}
\begin{itemize}
\item  \href{https://mycore.core-cloud.net/index.php/s/Es3xHEABEyf2uCv}{\small{\color{blue}https://mycore.core-cloud.net/index.php/s/Es3xHEABEyf2uCv}}
    \item SVG : Découpage des lettres et du background vectoriel (même si les lettres comme le background restent des images bitmap)
\item PNG: 4 tailles dispos
\item PDF : même propriété que le SVG, mais embarquable plus simplement en latex.
\item EMF : Même propriété que le SVG, mais embarquable plus simplement en powerpoint
\end{itemize}
\end{block}
}


\subsection{Site web}

\frame{
\frametitle{Site web mis en ligne}
\begin{block}{\href{https://dark-era.gitlabpages.inria.fr/projects/}{https://dark-era.gitlabpages.inria.fr/projects/}}
 \begin{itemize}
 \item Merci à François pour la mise en place du template Jekyll !
 \item Mise en page à améliorer
 \item Accès édition via \href{https://gitlab.inria.fr}{\textcolor{blue}{https://gitlab.inria.fr}} - onglet standard (creation compte -> Martin)
 \end{itemize}
\end{block}

\begin{block}{Les offres de stages/thèses/post-docs peuvent déjà y être postées !}
 \begin{itemize}
 \item Post-doc L2S (Automne 2021). \\
 \small{Draft \href{https://mycore.core-cloud.net/index.php/s/pqXVNgs0UbhthaB}{\textcolor{blue}{pdf}}}. Overleaf : \href{https://www.overleaf.com/7146167234xnpmfgbyfzdc}{\textcolor{blue}{https://www.overleaf.com/7146167234xnpmfgbyfzdc}}
 \item Stages L2S, IETR et IRISA (Printemps 2022)
 \item Thèses IETR et IRISA (Automne 2022)
 \item \textcolor{gray}{Post-doc Lagrange (Printemps 2023)}
 \end{itemize}
\end{block}

}


\frame{
\frametitle{Ok pour mettre sur le site web ?}
\includegraphics[width=11cm]{team_DARK-ERA}
}

\subsection{ISC High Performance conference}


\frame{
\frametitle{Session Poster à ISC High Performance conference}

\begin{block}{Poster version finale soumis}
\begin{itemize}
\item Attente de l'acceptation finale
\item \small{pdf : \href{https://mycore.core-cloud.net/index.php/s/z8RiX6jAq0JqOps}{\color{blue}https://mycore.core-cloud.net/index.php/s/z8RiX6jAq0JqOps}}
\item \small{overleaf : \href{https://www.overleaf.com/project/603a8bc405d5441e1ad25f7f}{\color{blue}https://www.overleaf.com/project/603a8bc405d5441e1ad25f7f}}
\end{itemize}
\end{block}



\begin{block}{En virtuel du 24 juin au 2 juillet}
\begin{itemize}
\item Session poster, Mercredi 30 juin (18h/19h heure de Paris) 
\item Video 5 mn, deadline 27 mai
\item Qui veut y participer ?
\end{itemize}
\end{block}
}


\subsection[Article CNRS]{Article pour revue CNRS sur interdisciplinarité}


\frame{
%\frametitle{Article pour revue CNRS sur interdisciplinarité}

\begin{block}{Article pour revue CNRS sur interdisciplinarité}
\begin{itemize}
\item SKALLAS invité par la MITI à rédiger un article (70 invitations sur 1000 projets)
\item Ouvrage collectif à destination du grand public (public averti mais non expert) 
\item Publication prévue chez CNRS-Éditions au courant de l’année 2021.
\item 4 pages, 12000 c. en français
\end{itemize}
\end{block}

\begin{block}{Planning de rédaction}
\begin{itemize}
\item Deadline initiale mi-mars pour la première version
\item \textbf{Version v0 à transmettre ce soir !}
\item Draft : \small{\href{https://upsud-my.sharepoint.com/:w:/g/personal/nicolas_gac_u-psud_fr/EbKT3LskAlFGq0bcSd1ACaABOzZqpEP1OzOAeu_I3X5hqA?e=iH1xVw}{\color{blue}https://upsud-my.sharepoint.com/:w:/g/personal/nicolas_gac_u-psud_fr/EbKT3LskAlFGq0bcSd1ACaABOzZqpEP1OzOAeu_I3X5hqA?e=iH1xVw}}
\end{itemize}
\end{block}


}


\subsection{Planning sessions Dark-era}


\frame{
\small{
\begin{block}{\textbf{\#1 Instruments} \hfill\textit{lundi 17 mai}}
\begin{tabular}{cl}
\textbf{9h/10h} &Séminaire Cyril+Cédric  \\
& \textit{Quelle est la nature et la taille des données d’entrée du SDP ?}\\
\textbf{10h/12h} &GT sur tâche \textbf{T1} \\
& \textit{\underline{Objectif} : définir des jeux de données et algo de réference}
\end{tabular}
\end{block}

\begin{block}{\textbf{\#2 SimGrid} \hfill\textit{mardi 1er juin}}
\begin{tabular}{cl}
\textbf{9h/10h} &Séminaire Martin+Fred  \\
& \textit{C'est quoi SimGrid ?} \\
\textbf{10h/12h} &GT sur tâche \textbf{T2} \\
& \textit{\underline{Objectif} : début des réflexions sur l'interface PREESM/SimGrid}
\end{tabular}
\end{block}

\begin{block}{\textbf{\#3 Algorithmes } \hfill\textit{vendredi 9 juillet}}
\begin{tabular}{cl}
\textbf{14h/15h} &Séminaire François+André  \\
& \textit{Algorithmes du SDP vu comme la résolution d'un problème inverse} \\
\textbf{15h/17h} &GT sur tâches \textbf{T3/T4} \\
& \textit{\underline{Objectif} : définir les familles d'algorithmes à étudier/prototyper}
\end{tabular}
\end{block}
}
}


\frame{
\frametitle{Sessions à programmer}

\small{
\begin{block}{\textbf{\#4 PREESM} \hfill\textit{sept/oct.}}
\begin{tabular}{cl}
\textbf{1h} &Séminaire Jeff+Karol+Mickael  \\
& \textit{C'est quoi PREESM ?} \\
\textbf{puis} &GT sur tâche \textbf{T2} \\
& \textit{\underline{Objectif} : poursuite des réflexions sur l'interface PREESM/SimGrid}
\end{tabular}
\end{block}

\begin{block}{\textbf{\#5 Accélération GPU/FPGA/MPPA} \hfill\textit{sept/oct.}}
\begin{tabular}{cl}
\textbf{1h} &Séminaire Nicolas+ Jeff(?) + Mickael(?)  \\
& \textit{Comment accélérer sur GPU/FPGA/MPPA les algos ?} \\
\textbf{puis} &GT sur tâche \textbf{T3} \\
& \textit{\underline{Objectif} : définir le \og cahier des charges \fg\  des prototypes} \\
\end{tabular}
\end{block}
}


}








\setcounter{tocdepth}{1}

\AtBeginSection[]
{
\frame[noframenumbering]{\tableofcontents[currentsection]}
}

\section{Séminaire Cédric+Cyril sur les \textbf{Instruments}}

%\subsection[Une méthodologie $A^3$ à petite échelle]{Une méthodologie $A^3$ à petite échelle \textit{{[}tomographie{]}}}


\section{GT benchmarking de SKAO}
 
 
\frame{
\frametitle{GT benchmarking de SKAO}

\begin{block}{Participation \textit{individuelle} au GT benchmarking}
    Chiara rappelle que la France a la responsabilité des équipements de calcul \textit{hardware}. Un Groupe de Travail sur le benchmarking au sein de la \textit{planet team} est en train de se constituer avec des ingénieurs d'institutions membres de SKA-France travaillant à mi-temps ou temps plein sur le sujet (INRIA, Obs. Cote d'Azur, Obs. Paris). Chiara encourage les chercheurs intéressés par ce GT à se manifester auprès d'elle.
\end{block}
    
\begin{block}{Lien entre Dark-Era et le GT benchmarking}
    Chiara préconise que le projet Dark-era se tienne régulièrement au courant des travaux de ce GT ; ce GT pourra être un moyen de  rendre visibles à la communauté SKA, la démarche et les premiers résultats de Dark-Era. Simon et Erwan qui y sont associés pourront nous transmettre les informations pertinentes pour Dark-Era. 
\end{block}
}






\section{Définition des jeux de données (tâche \textbf{T1}) }


\frame{
\frametitle{Jeux de données pour Dark-Era}

\small{
\begin{block}{Amorce de cahier des charges}
   \begin{itemize}
       \item La discussion sur les jeux de données a été brève.
       \item Un \textbf{wiki} (\textit{merci François pour la suggestion}) est mis en place sur le gitlab de dark-era pour y rédiger collectivement un cahier des charges des jeux données nécessaire à Dark-Era : \\
   \end{itemize}  
   
    \href{https://gitlab-research.centralesupelec.fr/dark-era/work/-/wikis/Cahier-des-charges-jeux-de-donnees}{\textcolor{blue}{https://gitlab-research.centralesupelec.fr/dark-era/work/-/wikis/Cahier-des-charges-jeux-de-donnees}}
\end{block}

}


}



\begin{comment}
\setbeamercolor{background canvas}{bg=black}
\section*{C'est fini !}
\frame[t,noframenumbering]
{

\vspace{3.6cm}
\centering
{\huge\textcolor{white}{Merci de votre attention}}

}
\end{comment}
\end{document}

