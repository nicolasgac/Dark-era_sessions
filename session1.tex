
\subsection{Logo}

\frame{
\frametitle{.. and the winner is :}
\includegraphics[width=0.49\textwidth]{DARKERA_logo_color.pdf} 
\includegraphics[width=0.49\textwidth]{DARKERA_logo_monochrome.pdf} 
\begin{block}{Proposé par Cédric tuné par Karol}
 clin d'œil cosmologique en rapport aux ages sombres (dark era) qui arrivent juste après le CMB (Cosmic Microwave Background).
\end{block}
}


\frame{
\frametitle{Logo disponible sous le cloud}
\begin{block}{Merci à Karol pour la déclinaison sous différents formats}
\begin{itemize}
\item  \href{https://mycore.core-cloud.net/index.php/s/Es3xHEABEyf2uCv}{\small{\color{blue}https://mycore.core-cloud.net/index.php/s/Es3xHEABEyf2uCv}}
    \item SVG : Découpage des lettres et du background vectoriel (même si les lettres comme le background restent des images bitmap)
\item PNG: 4 tailles dispos
\item PDF : même propriété que le SVG, mais embarquable plus simplement en latex.
\item EMF : Même propriété que le SVG, mais embarquable plus simplement en powerpoint
\end{itemize}
\end{block}
}


\subsection{Site web}

\frame{
\frametitle{Site web mis en ligne}
\begin{block}{\href{https://dark-era.pages.centralesupelec.fr/}{https://dark-era.pages.centralesupelec.fr/}}
 \begin{itemize}
    \item \textcolor{red}{Migration vers centralesupelec.fr}
 \item Merci à Martin pour la V0 initiale sur inria.fr et à François pour la mise en place du template Jekyll !
 \item Mise en page à améliorer
 \item Accès édition via \href{https://gitlab-reserach.centralesupelec.fr}{\textcolor{blue}{https://gitlab-reserach.centralesupelec.fr}} via l'onglet standard \\ \textit{comptes bientôt créés pour tous les membres de Dark-era}
 \end{itemize}
\end{block}

\begin{block}{Les offres de stages/thèses/post-docs à poster !}
 \begin{itemize}
 \item \textbf{Fait} : Post-doc L2S et thèse IETR 
 \item \textbf{A faire} : pour thèse IRISA (Automne 2022) + stages L2S, IETR et IRISA (Printemps 2022)
 \item \textcolor{gray}{Post-doc Lagrange (Printemps 2023)}
 \end{itemize}
\end{block}

}



\subsection{ISC High Performance conference}


\frame{
\frametitle{Session Poster à ISC High Performance conference}


\begin{block}{En virtuel du 24 juin au 2 juillet}
    \begin{itemize}
    \item Session poster, Mercredi 30 juin (18h/19h heure de Paris) 
    \item Poster + Video 5 mn, deadline 27 mai 
    \end{itemize}
    \end{block}

\begin{block}{Poster}
\begin{itemize}
\item \small{poster : \href{https://mycore.core-cloud.net/index.php/s/z8RiX6jAq0JqOps}{\color{blue}https://mycore.core-cloud.net/index.php/s/z8RiX6jAq0JqOps}}
\item \small{overleaf : \href{https://www.overleaf.com/project/603a8bc405d5441e1ad25f7f}{\color{blue}https://www.overleaf.com/project/603a8bc405d5441e1ad25f7f}}
\end{itemize}
\end{block}




}


\subsection[Article CNRS]{Article pour revue CNRS sur interdisciplinarité}


\frame{
%\frametitle{Article pour revue CNRS sur interdisciplinarité}

\begin{block}{Article pour revue CNRS sur interdisciplinarité}
\begin{itemize}
\item SKALLAS invité par la MITI à rédiger un article (70 invitations sur 1000 projets)
\item Ouvrage collectif à destination du grand public (public averti mais non expert) 
\item Publication prévue chez CNRS-Éditions au courant de l’année 2021.
\item 4 pages, 12000 c. en français
\end{itemize}
\end{block}

\begin{block}{Planning de rédaction}
\begin{itemize}
\item Deadline initiale mi-mars pour la première version
\item Draft : \small{\href{https://upsud-my.sharepoint.com/:w:/g/personal/nicolas\_gac\_u-psud\_fr/EbKT3LskAlFGq0bcSd1ACaABOzZqpEP1OzOAeu\_I3X5hqA?e=iH1xVw}{\color{blue}https://upsud-my.sharepoint.com/:w:/g/personal/nicolas\_gac\_u-psud\_fr/EbKT3LskAlFGq0bcSd1ACaABOzZqpEP1OzOAeu\_I3X5hqA?e=iH1xVw}}
\end{itemize}
\end{block}
}


\subsection{Planning sessions Dark-era}


\frame{
\small{
\begin{block}{\textbf{\#1 Instruments} \hfill\textit{lundi 17 mai}}
\begin{tabular}{cl}
\textbf{9h/10h} &Séminaire Cyril+Cédric  \\
& \textit{Quelle est la nature et la taille des données d’entrée du SDP ?}\\
\textbf{10h/12h} &GT sur tâche \textbf{T1} \\
& \textit{\underline{Objectif} : définir des jeux de données et algo de réference}
\end{tabular}
\end{block}

\begin{block}{\textbf{\#2 SimGrid} \hfill\textit{mardi 1er juin}}
\begin{tabular}{cl}
\textbf{9h/10h} &Séminaire Martin+Fred  \\
& \textit{C'est quoi SimGrid ?} \\
\textbf{10h/12h} &GT sur tâche \textbf{T2} \\
& \textit{\underline{Objectif} : début des réflexions sur l'interface PREESM/SimGrid}
\end{tabular}
\end{block}

\begin{block}{\textbf{\#3 Algorithmes } \hfill\textit{vendredi 9 juillet}}
\begin{tabular}{cl}
\textbf{14h/15h} &Séminaire François+André  \\
& \textit{Algorithmes du SDP vu comme la résolution d'un problème inverse} \\
\textbf{15h/17h} &GT sur tâches \textbf{T3/T4} \\
& \textit{\underline{Objectif} : définir les familles d'algorithmes à étudier/prototyper}
\end{tabular}
\end{block}
}
}


\frame{
\frametitle{Sessions à programmer}

\small{
\begin{block}{\textbf{\#4 PREESM} \hfill\textit{sept/oct.}}
\begin{tabular}{cl}
\textbf{1h} &Séminaire Jeff+Karol+Mickael  \\
& \textit{C'est quoi PREESM ?} \\
\textbf{puis} &GT sur tâche \textbf{T2} \\
& \textit{\underline{Objectif} : poursuite des réflexions sur l'interface PREESM/SimGrid}
\end{tabular}
\end{block}

\begin{block}{\textbf{\#5 Accélération GPU/FPGA/MPPA} \hfill\textit{sept/oct.}}
\begin{tabular}{cl}
\textbf{1h} &Séminaire Nicolas+ Jeff(?) + Mickael(?)  \\
& \textit{Comment accélérer sur GPU/FPGA/MPPA les algos ?} \\
\textbf{puis} &GT sur tâche \textbf{T3} \\
& \textit{\underline{Objectif} : définir le \og cahier des charges \fg\  des prototypes} \\
\end{tabular}
\end{block}
}


}






